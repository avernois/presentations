\def\raggedright{}
%\documentclass[mathserif,hyperref={pdfpagemode=FullScreen}]{beamer}
\documentclass[mathserif]{beamer}
%\documentclass[mathserif]{beamer}

\usetheme{340}

\usepackage[utf8]{inputenc}
\usepackage[francais]{babel}


\def\beginsectiontitle{Plan de l'expos\'{e}}

\def\bib[#1] {\smallskip\noindent\hangindent 10pt\hbox to10pt{#1\hfil}}

% %%%%%%%%%%%%%%%%%%%%%%% Convenient commands %%%%%%%%%%%%%%%%%%%%%%%
% \newcommand<>{\concept}[1]{{\color#2{red!75!green}#1}}
% \newcommand<>{\code}[1]{{\color#2{green!75!red!100!black!250}\texttt{#1}}}

%%%%%%%%%%%%%%%%%%%%%%% Packages utiles %%%%%%%%%%%%%%%%%%%%%%%%%%%%
%\usepackage[nocolor,noborder]{pdfswitch}
\usepackage{texgraphicx}
%\usepackage{alvingraphicx}
\graphicspath{{fig/}}
\usepackage{amsfonts,amsmath,amssymb}
\usepackage{enumerate}
\usepackage{url}\urlstyle{sf}
\usepackage{xspace}
\usepackage{relsize}
\usepackage{enumerate}
\usepackage{figlatex}
\usepackage{alvingraphicx}


\def\mbold{\mathversion{bold}}
\def\mnorm{\mathversion{normal}}
\def\clap#1{\hbox{{\setbox0=\vbox{#1}\copy0\kern-.5\wd0}}}
\def\Clap#1{\makebox(0,0)[b]{\smash{#1}}}
\def\L{\ensuremath{\mathcal{L}}\xspace}%
\def\R{\ensuremath{\mathcal{R}}\xspace}%

\newcommand{\cons}[1]{\ensuremath{\mathit{cons}(P_{#1})}\xspace}
\newcommand{\sent}[2]{\ensuremath{\mathit{sent}(P_{#1}\to P_{#2})}\xspace}
\newcommand{\link}[2]{\ensuremath{P_{#1}\rightarrow P_{#2}}}  
\newcommand{\A}[1]{\ensuremath{\mathcal{A}_{#1}}\xspace}  

%%%%%%%%%%%%%%%%%%%% des d\'{e}limiteurs sp\'{e}ciaux %%%%%%%%%%%%%%%%%%%%%%%%%%%
\let\leq=\leqslant
\let\geq=\geqslant
\let\epsilon=\varepsilon
\DeclareMathOperator{\argmin}{\text{argmin}}
\DeclareMathOperator{\argmax}{\text{argmax}}
\def\lb{\ensuremath{\llbracket}}  % gr\^{a}ce \`{a} stmaryrd
\def\rb{\ensuremath{\rrbracket}}  % gr\^{a}ce \`{a} stmaryrd
\def\clap#1{\hbox{{\setbox0=\vbox{#1}\copy0\kern-.5\wd0}}}

\def\smiley{\concept{\larger[3]\wasyfamily\char44}}
\def\frownie{\concept{\larger[3]\wasyfamily\char47}}

\newcommand<>{\blue}[1]{{\color#2{blue!100!black!100}#1}}
\newcommand<>{\red}[1]{{\color#2{red!75!red!100!black!100}#1}}
\newcommand<>{\green}[1]{{\color#2{green!70!black}#1}}

\newcommand<>{\lgray}[1]{{\color#2{white!70!black}#1}}

\newcommand<>{\lred}[1]{{\color#2{red!100!black!75}#1}}


\newcommand<>{\purple}[1]{{\color#2{blue!50!red}#1}} 


%%%%%%%%%%%%%%%%%%%%%%%%%%% Document %%%%%%%%%%%%%%%%%%%%%%%%%%%%%%%%%%
\hypersetup{%
 pdftitle={Coding Dojo},
 pdfauthor={Antoine Vernois}
}

\newcommand{\ignore}[1]{}


\title{Software Craftsmanship}

\author[antoine@crafting-labs.fr]{Antoine Vernois}
\date{5 janvier 2012}




\begin{document}

\maketitle

\frame{
    \frametitle{Constat}
    \begin{block}{}

        Si je veux apprendre le Judo, je vais m'inscrire au dojo du coin et y passer une heure par semaine pendant deux ans, au bout de quoi j'aurai peut-\^{e}tre envie de pratiquer plus assid\^{u}ment.
 
        Si je veux apprendre la programmation objet, mon employeur va me trouver une formation de trois jours \`{a} Java dans le catalogue 2004.

        Cherchez l'erreur.
        \newline
        \vfill -- \textit{Laurent Bossavit (2004)}
    \end{block} 

}

\frame{
    \frametitle{Un espace d'\'{e}change}
    \begin{block}{l'idée c'est}
        \begin{itemize}
            
            \item un lieu de partage
            \item un lieu d'exp\'{e}rimentation
            \item un lieu de pratique
            \item un lieu d'apprentissage
        \end{itemize}
    \end{block} 
    
    \begin{block}{mais \c{c}a n'est pas}
        \begin{itemize}
            \item un prof qui dispense la bonne parole
            \item une formation passive
        \end{itemize}
    \end{block} 
}

\frame{
    \frametitle{les th\`{e}mes abord\'{e}s}
    \begin{block}{des techniques}
        \begin{itemize}
            \item des algos
            \item des patterns
            \item de la conception
        \end{itemize}
    \end{block}
    \begin{block}{des pratiques}
        \begin{itemize}
            \item pair programing
            \item TDD
            \item BDD, ATDD
            \item conception \'{e}mergente
        \end{itemize}
    \end{block}
}

\frame{
    \frametitle{des formats}

    \begin{block}{}
        
	    \begin{itemize}
		\item coding dojo
                \item code retreat
                \item courtes présentations
                \item code \& breakfast
                \item ... 
	    \end{itemize}
    \end{block}
}


\frame{
    \frametitle{y'a plus qu'\`{a} !}
    \begin{block}{}
       \center{ Y'a plus qu'\`{a} !}
    \end{block}
}

\end{document}
